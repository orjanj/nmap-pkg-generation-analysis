\section{Introduction}
\label{s:ConclusionIntroduction}
In this thesis, the development of a framework for synthetic packet generation and characterization of Nmap scans are discussed.
During this thesis, a series of tests have been conducted while continuously developing both the scripts needed to conduct the automated scanning procedures together with the other necessary components for task management.
After the lab environment was completed, the packet generation was started. Issues with the developed tools were immediately fixed when encountering errors making this environment more reliable.
The given scan numbers were 10 for each scan type and 10 for each timing template, resulting in 60 scans for one scan type. In total, 360 packet captures were created for empirical analysis purposes.
When all tasks were completed and the number of given scans was reached, the development of the analysis template in Jupyter was continued. This was a work in parallel while waiting for all scans to be completed.
To be able to analyze the captured datasets, the raw packet captures had to be parsed to a comma-separated value format.
The development of the \textsc{packet capture parser} was done. The parser converted the raw packet captures to CSV files, and the analysis could start.
Giving the required parameters to the analysis template, a run through the given Jupyter notebook was completed resulting invaluable data and visualizations later used for analysis, described both in chapter \ref{chp:DataAnalysis} and chapter \ref{chp:Results}.




