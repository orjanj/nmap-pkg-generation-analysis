\section{Scanning activity and security audit towards own network}
\label{s:ScanOwnNetwork}
A system administrator could implement network scanning towards its network to identify alive hosts within the network, figuring out identification of operating system in use and service information. A vulnerability scan could be conducted to identify weak spots within the network, like a comparison between gathered service information and a vulnerability database. This is often conducted during an audit of a network that aims to identify vulnerabilities without interfering with running services. This method is called vulnerability analysis. The gathered information could be further used in the evaluation of the vulnerable parts of the network. The assessments during the network and vulnerability scan base the signatures on which operating system is used, running services and vulnerabilities \autocite{holm2011quantitative}.
The popular scanner \hyperref[s:HowNmapWorks]{Nmap} are widely used by security professionals during compliance testing, asset management, and system inventory. It has multiple usage areas such as verifying firewall filters, host discovery, policy compliance, and system monitor uptime \autocite{pinkard2008nmap}.
