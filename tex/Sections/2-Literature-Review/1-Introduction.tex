\section{Introduction}
In this chapter the literature review is presented together with background resources for the research.

Scanning attacks are often applied to collect information about a network, mainly in the first step of the Cyber Kill Chain (CKC) known as the reconnaissance phase. The Lockheed Martin Cyber Kill Chain is a framework used for identification and prevention of cyber intrusions activity, which is a part of the Intelligence Driven Defense model \autocite{lockheedmartinCKC}. The framework describes seven phases of a kill chain to accomplish a successful attack. The first step of the CKC is reconnaissance involving identification of targets such as gathering email addresses and social relationships for important persons in a targeted company, running services or other technical information relevant to moving on with a potential cyber attack \autocite{hutchins2011intelligence}. Applying mitigation actions within the steps in the CKC framework proves to be efficient since during a cyber attack all seven phases needs to be successful to allow a cyber attack to occur \autocite{lockheedmartinCKC}.

During the phase of intelligence gathering, three large steps exists; foot printing, scanning and enumeration \autocite{arkin1999network}. These are methods used for intelligence gathering and used as an initial step to gain access to a network. Other malicious activities such as scanning in a paper by \textcite{Treurniet2011} argues that it is difficult to detect scanning since it's neither well-behaved nor well-understood. Insight can be given by classifying known activities, giving visibility to threats and extending awareness of network activities \autocite{Treurniet2011}.

Scanning are also a central part of a penetration testing, where vulnerabilities are identified. During the phase of scanning and information gathering activities such as port and vulnerability scanning are conducted, as a normal thing for gathering intelligence. This proves the importance of detecting potential cyber attacks in the early stage by applying early warning such as scanning detection and categorisation.

Popular tools conducting penetration testing is the large open-source project Metasploit, which comes default with a large database of already tested exploits, and is currently one of the biggest open-source projects within the field of penetration testing and information security \autocite{MetasploitBook}. Metasploit is further described in section \ref{s:HowMetasploitWorks}.
The port scanning tool Network Mapper (nmap), elaborated in section \ref{s:HowNmapWorks}, is a useful network scanning tool for information gathering involving targets running services and open ports \autocite{pinkard2008nmap}.
Nessus is a tool used for vulnerability assessments, described in section \ref{s:HowNessusWorks}.

Within the sections in the literature review the historical development are presented, various types of scanners are elaborated, how research have classified network scans, how network scan works and how to detect it.