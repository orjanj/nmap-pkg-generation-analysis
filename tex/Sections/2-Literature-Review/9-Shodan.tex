\section{Shodan}
\label{s:HowShodanWorks}
Shodan is the result of a idea from John Matherly. It's used as a search engine looking for devices connected to the internet. The primarily part of data that Shodan collects is the banner result from a scanned host. Information given back to Shodan after a scan where a port is open, could return service version information and other specifics for the given service \autocite{ShodanGuide}.

In the late 2009, John Matherly asked friends trying out his new project called Shodan, an idea conceived in 2003. The goal of the project were to map all devices connected to the internet. The results from the project were extraordinary, and Shodan shown it's possibilities to find water plans, power grids, the cyclotron at Lawrence Berkeley National Laboratory, SCADA systems and thousands of unsecured Cisco equipment among other things. The scanning service has shown useful for security researchers, threat actors and hackers since the results of a scan could be used to gain unauthorised access to critical systems. Some say that barely no competence is needed to take advantage of the results if the targeted host has not protected their systems good enough started working on a project named Shodan, where the goal was to map out devices connected to the internet \autocite{Shodan2012}.