\section{The historical development of network scanners}
\label{s:HistoryScanning}

On the 2nd November 1988, a worm with the name of Morris Worm was created by Robert Morris. The main objective was to steal military information. The result out of this worm was a conviction against Robert Morris, who targeted 6.000 computers on the Internet. Morris was the first person ever getting convicted for breaking the law of gaining unauthorized access to protected computers. This led to the creation of what we today know as a Computer Emergency Response Team \autocite{DayaBhavya,FBIMorris}.

Later in the 1990s, Kevin Mitnick committed several crimes, where tools such as scanners were a central part of the activities committed. Other persons included in these activities were Zyklon Burns, Patrick Gregory, and Chad Davis, where Burns received 15 months in prison, while the others received 26 months. During a hack against the White House, Zyklon initially used a CGI scanner, which especially targets vulnerabilities for common gateway interface. In the book, a port scanner is mentioned as a central breaking-in point to gather intelligence about the open ports during a bank break-in \autocite{MitnickArtIntrusion}. Mitnick used \hyperref[s:HowNmapWorks]{nmap} to scan for all open ports during an intrusion later in London, where the network were mapped, identified a router and open ports \autocite{MitnickArtIntrusion}, which proves that nmap fast grew becoming a popular tool.

On 27th February 1995, Julian Assange initially released a version of the TCP port scanner Strobe \autocite{StrobeAssange}. Strobe evaluates listening TCP ports on hosts segmenting bandwidth between hosts in an efficient matter \autocite{StrobeMan} and finds open ports on a system \autocite{JulianAssangeBook,BLYTH199943}.
Julian Assange had earlier a history of hackings taken place, where he on the 5th December 1996 pleaded guilty for hacking Nortel, where he was sentenced to three years of probation and a fine on \$2.300 \autocite{JulianAssangeBook}.

In September 1997 the first release of the port scanner nmap was released. Nmap was mainly created to conduct an audit of systems for system administrators to keep one step ahead of hackers. It has a lot of features created to perform scans from basic ICMP pings to check for alive hosts, to conducting operating system detection, and it also supports TCP and UDP \autocite{pinkard2008nmap}.

The history moves on, and other tools with the main purpose of conducting port scanning were created such as NetScanTools, Superscan, Angry IP Scanner, and Unicornscan. The working area and feature description of these tools would be further described in \hyperref[s:HowToScanning]{section 2.4}. These tools were initially released within the period of 1998 and 2006.

During 1998, the large focus on network security grow. At the time the Security Administrator Tool for Analyzing Networks (SATAN) were shown to be obsolete and other commercial scanners shown to be more comprehensive. \hyperref[s:HowNessusWorks]{Nessus} were created in 1998, growing from a small community to a community with 149 participants and 114 plugin writers in 2008. During a licence change of \hyperref[s:HowNessusWorks]{Nessus} version 3 at 5th October 2005 changing from open-source to proprietary multiple forks where created as a consequence. These two projects are Porz-Wahn and OpenVAS \autocite{rogers2011nessus}.

In the late 2009, John Matherly asked friends trying out his new project called Shodan, an idea conceived in 2003. The goal of the project were to map all devices connected to the internet. The results from the project were extraordinary, and Shodan shown it's possibilities to find water plans, power grids, the cyclotron at Lawrence Berkeley National Laboratory, SCADA systems and thousands of unsecured Cisco equipment among other things. The scanning service has shown useful for security researchers, threat actors and hackers since the results of a scan could be used to gain unauthorised access to critical systems. Some say that barely no competence is needed to take advantage of the results if the targeted host has not protected their systems good enough started working on a project named Shodan, where the goal was to map out devices connected to the internet \autocite{Shodan2012}.