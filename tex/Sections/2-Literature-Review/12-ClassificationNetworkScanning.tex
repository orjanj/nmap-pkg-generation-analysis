\section{Classification of network scanning activities}
\label{s:ClassificationOfScanning}
In the cyber space today threats have become a major concern for businesses. A paper by \textcite{fekolkin2015intrusion} discusses what is IDS and IPS and which security measures needed to be implemented other than anti-virus and firewall. In network intrusion a network scan is a common reconnaissance activity \autocite{TaxonomyOfNetworkScanning}. The paper further describes a classification of network scanning and also illustrates the complexity of the activity. \textcite{Liu2014TowardsAT} shows that over 96\% of all anomalous events can be detected and labelled. This results would make unlabelled source rate very low.

For the mission of detecting abnormal activity within a network, a taxonomy of anomalies could be created to detect such behaviour. There exist multiple techniques for detection, such as creating signatures for pattern recognition, statistical methods, among other things. These methods provide information that further could be used for anomaly detection. Traffic metrics could be used for detecting anomalies, such as scanning activity and DDoS. Labelling each occurring event could be applied to categorise and differ between abnormal traffic and normal traffic. Categorisation of traffic could be separated into subcategories such as unknown, anomalies, or normal \autocite{6906328}.

Scans are one type of event among others, which could be represented through characteristics and patterns. It might be several hosts or single machines that are scanned. Scans are conducted to gain knowledge about a target. This can be opened ports, which operating system running, or versions of software that are using the open ports. In a network scan, hosts alive could be detected, among services and ports open. During a host scan, the same results could be retrieved, through checks towards the allowance of packets to a specific port. Network scans can be used to gather intelligence of how the network is structured, how many clients are connected, and which services at which version are located on the network. After gathering intelligence about clients in the network, the intelligence itself can tell something about which exploitation could be used against the various services at which specific versions \autocite{6906328}.

Various traffic characteristics during scanning would be caught, such as the following. During TCP scanning, SYN is used to initiate connection while ACK is mapping ports by filtering port answers through ICMP \autocite{6906328}.
A signature is created with rules for different attributes of the traffic. Abnormal characteristics such as scanning attempts can be detected and generated a signature out of the behaviour of the traffic recently conducted \autocite{6906328}.


While identifying scanning techniques, a quantitative amount of data is required to conduct identification and evaluation of scans conducted. During the research of \textcite{TaxonomyOfNetworkScanning} the port scanning tool \hyperref[s:HowNmapWorks]{nmap} was used to construct the scanning, and tcpdump was used to capture packets on the network interface of a FreeBSD host. Scripts were constructed to speed up the analysis and be able to analyze one month of data in about five minutes. 
