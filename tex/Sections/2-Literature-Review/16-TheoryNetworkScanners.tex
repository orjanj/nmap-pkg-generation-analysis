\section{Theory behind network scanners and exploration of scanning methods} % TODO: Review title!
\label{s:HowToScanning}

Network scanning is a reconnaissance activity conducted within the field of network intrusion \autocite{TaxonomyOfNetworkScanning}. During an intelligence-gathering step, network scanning is one of the first steps before an attack is about to happen \autocite{HOUMZ2021103230}.
Conducting a network scan would reveal information about hosts on a network, which includes running services and applications, open ports, and operating system details. The main techniques applied during network scanning are version and service detection, scanning ports, operating system detection, and network mapping \autocite{pinkard2008nmap}. During a network scan devices and services vulnerable for attacks could be unveiled \autocite{HOUMZ2021103230}.


%------------------ PORT SCAN ------------------ 
\subsection{Port scan and service scanning}
\label{s:PortScanning}
\label{s:ServiceScan}

By conducting port scanning, information about running services and open ports could be retrieved.
This could further be used to determine misconfiguration on a system or version of the software which could be mapped against known exploitable vulnerabilities. It exists multiple types of port scanning methods, further elaborated below \autocite{pinkard2008nmap}. Port scanning is a reconnaissance technique used to discover services running on open ports which could be exploited to gain access to a system. Port scanning is primarily divided into vertically and horizontally scans, further elaborated in section \ref{s:HorizontalScanning}. In vertical attacks, described in section \ref{s:VerticalScanning}, multiple ports are scanned on one host. This is a common process when a threat actor is interested in one specific host to attack \autocite{dabbagh2011slow}.
%------------------ ICMP ECHO REQUESTS ------------------ 
\subsection{Host discovery using ICMP echo requests (ping scan)}
\label{s:ICMPEcho}

ICMP echo requests could be used for the purpose of determining whether a host is online or offline.
According to the RFC 792 \autocite{rfc792} it exists multiple types of ICMP types which are messages for a request or reply message, such as timestamp, redirect, echo, timestamp, information, source quench, destination unreachable and time exceeded.
The echo type, type number 8 is a echo request message sent to a target to determine if an reply could be received. The echo reply message has type number 0. The ICMP echo requests can be used for determine whether an host is online or not \autocite{rfc792, 10.5555/1538595}.
Many firewalls have by default configured to deny echo ICMP requests \autocite{arkin1999network}.
The nmap scanner can conduct this sort of operation by using the parameters $-PE$ in a scan \autocite{10.5555/1538595}.
%------------------ STEALTH SCAN ------------------ 
\subsection{Stealth scans}
\label{s:StealthScan}
A stealth scan is what the name suggests it to be, to be a scan that should not be detected.
Some of the evasion techniques used for a stealth scan is using various flag settings and fragmentation.
Fragmentation means splitting up the scan requests over multiple packets by attempting to not getting detected while scanning. This applies to the TCP-based scanning requests \autocite{pinkard2008nmap}.

Other evasion techniques such as slow and low scanning can be applied in order to not getting detected by an intrusion detection system (IDS). The technique limits the number of ports targeted for a larger timespan than a fast scan. By reducing the time of a scan reduces the risk of being detected. Some of the stealthy scanning techniques is NULL scan, XMAS scan, FIN scan, SYN/ACK scan and ACK scan \autocite{pinkard2008nmap}.


%------------------ XMAS SCAN ------------------ 
\subsubsection{XMAS Scan}
\label{s:XMASScan}
A Nmap XMAS scan sets the FIN (finish), PSH (push) and URG (urgent) flag during a scan. These flags should not occur in normal traffic and are therefore called invalid flags. The conventions for these types of packets are not established meaning various stacks in TCP would respond differently. Some TCP stacks would return a RST (reset) packet from both open and closed ports, while other systems can return no reply at all. The most typical is a reply with RST/ACK (reset/acknowledge) for closed ports, while open ports not replies at all and drops the received packet.
The parameter for conducting a TCP XMAS Scan in nmap is $-sX$ \autocite{pinkard2008nmap, 10.5555/1538595}.
The XMAS scan can based on this determine open and closed ports.
%------------------ /XMAS SCAN ------------------ 



%------------------ NULL SCAN ------------------ 
\subsubsection{Null Scan}
\label{s:NullScan}
A nmap null scan does not set any flags, it sets the flag header to $0$. The name null scan abbreviates from this.
Nmap null scan has the ability of being stealthy by avoiding intrusion detection systems, firewalls and system logging.
A open port will, like the XMAS scan, drop the received packet and not reply to the scan request. Meanwhile, an closed port replies with RST/ACK (reset/acknowledge), where some systems replies with RST (reset) for all ports, and other systems might not reply at all to any received packets \autocite{pinkard2008nmap, 10.5555/1538595}. This is similar to the XMAS scan, elaborated in section \ref{s:XMASScan}. The parameter for conducting a null scan in nmap is $-sN$ \autocite{pinkard2008nmap}.


%------------------ FIN SCAN ------------------ 
\subsubsection{FIN Scan}
\label{s:FINScan}
The FIN scan are similar to the XMAS scan described in section \ref{s:XMASScan} and the Null scan described in section \ref{s:NullScan}. Though, it only sets the FIN (finish) flag during a scan compared to the two mentioned scanning methods. A port is considered closed if a RST (reset) is given in the reply. Considered as a open port is no reply \autocite{pinkard2008nmap, 10.5555/1538595}.
The parameter for FIN scan in nmap is $-sF$.
%------------------ /FIN SCAN ------------------ 
%------------------  CONNECT SCAN ------------------ 
\subsection{Connect scan}
\label{s:ConnectScan}
During a connect scan, a three-way handshake is performed to open a connection to the given target.
The connect scan is abbreviated in this research to \textit{TCP full scan}.
A regular reply would be an SYN/ACK (synchronize/acknowledge) packet if a TCP port is listening on the given port, otherwise, a host would respond with an RST/ACK (reset/acknowledge) \autocite{pinkard2008nmap, dabbagh2011slow}.
This depends also on other factors such as the firewall configurations, where hosts would reply with RST/ACK if the firewall blocks the connection. TCP connect scans have a higher likelihood to be identified and logged compared to stealth scans, described in section \ref{s:StealthScan}. By using the Nmap connect scan, Nmap uses the operating system for connection establishment rather than customising the raw packet when sending it to a target. This type of scan creates a large number of connection attempts to various target ports, and this often indicates a ongoing port scanning. Due to this, the risk of being detected is significantly higher than other types of scans. Systems that do not close TCP connections in a proper manner can suffer from denial of service (DoS) conditions. The parameter to use in Nmap for conducting a connect scan is $-sT$ \autocite{pinkard2008nmap}.