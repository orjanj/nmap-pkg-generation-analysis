\section{Network Mapper (nmap)}
\label{s:HowNmapWorks}
Network mapper (nmap) is free software released in September 1997, mainly created for use to scan networks under security audits by system administrators and other IT professionals.
It has a wide range of capabilities, such as scanning single hosts for detecting the versioning of software running on open ports. Another capability is creating IP packets whose utilised to perform more specific scans towards a single or multiple hosts. The software has various usage areas, such as asset management, system, and network inventory, compliance testing, and security auditing. During audits, it could be useful to conduct OS versioning to detect the need for patching and updates. Since nmap could usually be used for this purpose, it could also be used during harmful intrusions to map the infrastructure of a network. Furthermore, this could be used to exploit vulnerabilities within the network.

Nmap is free and runs on multiple operating systems, most used on Windows, Linux, and MacOSX.
Regarding types of scan that could be conducted in nmap, section \ref{s:HowToScanning} describes the various ways of behaviour for scans. Nmap have a lot of scanning capabilities, such as ICMP echo request, stealth scan, XMAS scan, null scan and connect scan \autocite{pinkard2008nmap}.
