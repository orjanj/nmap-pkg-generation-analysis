\subsection{Packet capture retrieval}
\label{ss:PcapRetrieval}

This script is created for easier retrieval of packet captures from each worker host.
During the initial research, packet captures, including noise traffic, were generated.
Within this script, a filter is applied to reduce the retrieval of these packet captures.
During the research, this filter mainly consisted of a timestamp starting with year and month to retrieve the correct files.
This script uses rsync to retrieve packet captures from each worker host.
\begin{listing}[!ht]
\caption{Retrieving packet captures from worker hosts}
\label{lst:PcapRetrival}
\begin{minted}{Bash}
#!/bin/bash
WORKER_HOSTS=(bsc01-mng bsc02-mng bsc03-mng bsc04-mng bsc05-mng bsc06-mng bsc07-mng 
bsc08-mng bsc09-mng bsc10-mng bsc11-mng bsc12-mng bsc13-mng bsc14-mng bsc15-mng 
bsc16-mng bsc17-mng bsc18-mng bsc19-mng bsc20-mng)
PCAP_FILTER_NAME=$1
OUTPUT_DIRECTORY=$2

if [[ -z $PCAP_FILTER_NAME ]] || [[ -z $OUTPUT_DIRECTORY ]]; then
  echo "Usage: $0 <pcap file filter name> <sync destination>"
  exit
fi

for WORKER_HOST in ${WORKER_HOSTS[@]};
do
  rsync -azvv -e ssh "bscadm@$WORKER_HOST:$PCAP_FILTER_NAME*.pcap" $OUTPUT_DIRECTORY/
done
\end{minted}
\end{listing}
