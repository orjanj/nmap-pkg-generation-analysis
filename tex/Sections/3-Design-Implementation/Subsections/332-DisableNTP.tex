%% ---------------- NTP ----------------
\subsection{Disabling NTP}
\label{ss:DisableNTP}
Default in a Ubuntu distribution, like the 20.04 used in this research, a time and date service is active for the purpose of keeping the system updated with the correct date and time.
The timedatectl would generate noise traffic such as NTP requests to ntp.ubuntu.com for synchronizing the time.
During the initial setup, before noise reduction measurements were taken, a significant number of NTP requests were sent from the worker in order to reach ntp.ubuntu.com.
The following command was executed to disable the network time
synchronization\footnote{\OrjansHref{https://www.man7.org/linux/man-pages/man1/timedatectl.1.html}{https://www.man7.org/linux/man-pages/man1/timedatectl.1.html}}.

\begin{listing}[!ht]
\caption{Command for disabling NTP synchronisation}
\label{lst:CommandDisableNTPSync}
\begin{minted}{Bash}
timedatectl set-ntp 0
\end{minted}
\end{listing}

This command stops the synchronisation services which enables the virtual machine to use the host's clock.