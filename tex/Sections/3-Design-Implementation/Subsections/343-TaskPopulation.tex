\subsection{Task population}
\label{s:TaskPopulation}
A script was developed for populating a unique tasklist for streamlining the process of creating tasks.
The script reads one input parameter, which is a template task file created by a user.
This input file is then iterated through, and an incrementing number is added to each task for the identification of unique tasks later during a scan.
\begin{listing}[!ht]
\caption{Populating tasks to task file}
\label{lst:TaskPopulation}
\begin{minted}{Bash}
#!/bin/bash
INPUT_FILE=$1

if [[ -z $INPUT_FILE ]]; then
  echo "Usage: $0 <task-file>"
  echo "Default number of task param is \"50\" - must be changed in for loop in line 12"
  exit
fi

while IFS=, read -r PRIORITY TASK_NAME TASK_STATUS SCANNER EXTRA_ARGS
do
  for i in {1..50}
  do
    echo "${PRIORITY},${TASK_NAME}_${i},${TASK_STATUS},${SCANNER},${EXTRA_ARGS}"
  done
done < $INPUT_FILE
\end{minted}
\end{listing}

To populate a new task list file, the following command needs to be executed in a Bash shell.
\begin{minted}{Bash}
user@host:~# bash populate.sh template-taskfile.csv >> new-taskfile.csv
\end{minted}
This will append each line generated to the `new-taskfile.csv`.