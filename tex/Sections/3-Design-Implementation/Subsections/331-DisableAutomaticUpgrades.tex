%% ---------------- SNAP PACKET MANAGER ----------------
\subsection{Disabling automatic upgrades and stop package managers}
\label{ss:DisableSnapAPT}
Default delivered in Ubuntu Linux 20.04 is the older APT (Advanced Package Tool) and the newer Snap package manager. Packages delivered through the \textsc{Snap package manager} are called snaps.
There exist multiple methods of package management in Ubuntu, such as using Synaptic Package Manager, Ubuntu Software, apt, apt-get, or snap. It is also possible to download raw deb files and install them using the pkg tool. In the Ubuntu Software in Ubuntu 20.04, Snap is prioritized when installing packages in front of the commonly used APT package manager. DEB (Debian package format) packages are the older format for packages in Ubuntu Linux, which APT uses. Snaps are formatted differently and cannot be installed through APT but instead installed through the Snap package manager. The Snap package manager uses snapd daemon for package management. The Snap package format will replace the older DEB package in Ubuntu \autocite{petersen2020ubuntu}.

Noise traffic generated by the automatic update polling for the package managers can be reduced. A few configuration parameter changes must be implemented to disable the automatic updates.
Disabling automatic updates like this is a security risk for systems connected either to the internet or a local network. The workers are segmented on their own separated virtual network on one virtual host and not directly connected either to the internet or the local network. Therefore the risk during this research is acceptable.
By changing the given parameters for the auto upgrades seen in listing \ref{lst:AutoUpgradesApt} from $1$ to $0$ the automatic upgrades for the APT package manager are disabled.




\begin{listing}[!ht]
\caption{Disabling automatic updates for APT}
\label{lst:AutoUpgradesApt}
\begin{minted}{perl}
# File contents for /etc/apt/apt.conf.d/20auto-upgrades
APT::Periodic::Update-Package-Lists "0";
APT::Periodic::Unattended-Upgrade "0";
\end{minted}
\end{listing}
The snap package manager has a number of services running in Ubuntu 20.04 by default.
These services can be disabled by running two commands for each service.
The $systemctl$ $stop$ command stops the running service, while $systemctl$ $disable$ disables the service from running on startups. The commands in listing \ref{lst:DisableSnapServices} must be run as $root$ on the worker host in the Bash shell command prompt.
\begin{listing}[!ht]
\caption{Disabling Snap services}
\label{lst:DisableSnapServices}
\begin{minted}{Bash}
systemctl stop snapd
systemctl disable snapd
systemctl stop snapd.socket
systemctl disable snapd.socket
systemctl stop snapd.service
systemctl disable snapd.service
systemctl stop snapd.seeded
systemctl disable snapd.seeded
systemctl stop snapd.snap-repair.timer
systemctl disable snapd.snap-repair.timer
systemctl stop snapd.apparmor
systemctl disable snapd.apparmor
\end{minted}
\end{listing}

By implementing these changes the automatic updates will be disabled both for the APT package manager and the Snap package manager, and it will reduce the noise traffic generated.