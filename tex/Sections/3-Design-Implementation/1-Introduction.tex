\section{Introduction}
\label{s:DesignImplementationIntroduction}
This chapter presents the overview of the design, together with the technical implementation of the lab environment and code used to conduct the data generation.

The importance of automatically generating synthetic packet captures is further discussed in this chapter.
Having a standard of conducting each network scan enhances the reliability of each scan being conducted in a similar standardized way compared to manually running these Nmap scans, where differences might appear depending on the user inputs while executing the commands and human error.
In cases where this is conducted manually, the risk of not standardizing the scans can lead to incomparable data as an output during the analysis phase.
Therefore, the importance of automating the scans with a standard is crucial. The automation is also time-saving compared to a manual scanning procedure.
This automation is described in this chapter, where the main components are described. Among these is the important scanner script elaborated in section \ref{s:ScannerScript}.
By the use of various timing templates, the workers finish the scans at different times, which makes a manual scanning method very time-consuming and not effective. This design and implementation chapter will describe the setup of such an automation packet capture generation.
