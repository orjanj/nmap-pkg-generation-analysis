\section{Introduction}
Within this chapter, the results conducted through the analysis earlier discussed in section \ref{s:DataGAAnalysis} be presented and discussed.
By using the Jupyter notebook for conducting the analysis, visualizations are created in order to clearly see patterns that require further analysis.
The most significant findings during the analysis are presented within this chapter.

Chosen Nmap scanning types are \textit{xmas}, \textit{connect} (TCP full scan), \textit{null}, \textit{fin}, \textit{host discovery (ping)} and \textit{service scan}, further described in section \ref{s:HowToScanning}.
All of Nmap's timing templates were used in the data generation, which is used for comparing the various scanning types in this chapter.
To create a larger basis for comparison when visualizing diagrams in the results, one can type selected together mixed with one timing template.
In this case, both the timing templates and the scan types can indicate what is normal for a scan by using visualizations.

Packet count metrics are further elaborated based on the selection process mentioned above.
The metrics extracted are the packets per second (PPS) indicating standard deviation, minimum and maximum packets per second.

Port sequence analysis was applied in order to gain an understanding of the eventual existence of the same port sequence for each separate scan.
This will be further described in the chapter, together with a number of unique source ports used for each scan.
Each packet number is plotted in a diagram to visualize where the main emphasis of ports is targeted.
Furthermore is the usage of both targeted (abbreviated as \textit{used ports}) and untargeted (abbreviated as \textit{unused ports}) ports presented.

