\begin{abstract}
Today the increased use and peoples dependence on the internet have resulted in a significant attack surface for threat actors around the world.
Scanning is the most important activity regarding information gathering for use in potential cyber attacks.
People today use network scanning for multiple purposes; conducting audit towards own manageable networks, researching purposes or conducting cyber attacks, and during autonomous malicious activity.
It's important to clearly have a base knowledge with how a network scanner works, which various types of scanning can be conducted and how this can be identified.
Related to pending security action being directed, a base knowledge of which types of tools to conduct various tasks such as audit is crucial to mitigate risks of network scanning.
Characterising network activity is a important step of mitigating future risks of cyber attacks.
A very important thing is to detect such activity, and it exists multiple mechanisms that can be applied.
This thesis will describe the framework for synthetic packet generation and characterization of Nmap scans, together with the evolution and historical background of network scanning methods from the early internet age until today. The framework includes tools for automating generation and processing of packets, together with templates for conducting empirical analysis.
\keywords{Network scanning, NMAP, network enumeration, detection, scanning}

\end{abstract}
