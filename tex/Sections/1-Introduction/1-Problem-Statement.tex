\section{Problem Statement}

A system administrator would have full control of a network, and network scanning is a large proportion of noise towards networks all around.
Today there exists intrusion detection systems capable of detecting such activity and firewalls for blocking malicious network traffic.
Network scanning is a step of collecting intelligence and information about a target regarding open and locked ports and is considered the first step in Lockheed Martin's framework Cyber Kill Chain \autocite{lockheedmartinCKC}.
Reconnaissance activities are important to gain knowledge of potential ways of breaking into a system or network.
These threats should ideally be detected and mitigated within the first step of the Cyber Kill Chain.

Not only do threat actors use network scanning, but penetration testers and system administrators also use network scanning as a tool for mapping out exposed services that either can be poorly configured or services running by default after an operating system installation.
These arguments emphasize the importance of having proper intrusion detection systems in place, hardened systems, and strictly configured systems.
Important knowledge for implementing rules and configurations to such systems is to understand patterns and threats of a cyber attack to enable early detection of these threats.
An important step to accomplish this is to primarily detect widely known scanners as an important step for both ongoing scanning activity and prevention of cyberattacks.
Understanding of these patterns can be achieved with empirical analysis.

